%___________________________________________________________________________________________________________________
% Refer to:
%   https://github.com/pmarcum/WorPT-Work-Plan-Tool-4-proposals/blob/main/latexTableTemplates/[INSERTFILENAME]/readme.md
% for instructions and LaTeX code to copy/paste into your document, to incorporate this file. 
%___________________________________________________________________________________________________________________

\def\WorPTfolder{[INSERTFILENAME]} % set the folder name where WorPT files are contained

% ___________________________________________
%       Default formatting preferences
% ___________________________________________

%---- appearance of top of each biosketch showing colorized name and title underneath 
\def\NameFontstyle#1{\textbf{#1}}              % boldface name/role
\def\NameColor{Blue}                          % font color of name/role
\def\NameSize{\large }                        % font size of name/role

%---- formatting of the font in the labels such as "Education", "Appointments", etc.
\def\LabelFontstyle#1{\textbf{#1}}             % boldface "Education",... labels

%---- vertical spacing between the "Education", "Appointments", etc. categories
\def\SectionSpacing{\par \vspace{-0.5em}}     % vertical spacing bet/ categories

%---- in the publication list, defines the symbol appearing at the start of each pub
\def\PubSym{\scriptsize{$\bullet$}{\hspace{0.5em}}}

\newenvironment{[INSERTFILENAME]}{%

[INSERTBIOSKETCHLIST]

% COMMENT: the rest of this file serves as a template for each person's bio sketch.
% COMMENT: The bio sketches are then "glued together" by either \newpage or carriage returns
% COMMENT: and then put into the INSERTBIOSKETCHLIST above. 

% COMMENT: below provides global format for a bio sketch report
[STARTPERPERSONTEMPLATE]
% generate the name label at the top using the defined formatting, put affiliation below
\NameFontstyle{\color{\NameColor}{\NameSize [INSERTNAME] ([INSERTROLE]):}}\\
[INSERTAFFILIATION]
% make EDUCATION section
\SectionSpacing
\LabelFontstyle{Education}\\
[INSERTEDUCATIONLIST]
% make the APPOINTMENTS section
\SectionSpacing
\LabelFontstyle{Appointments}\\
[INSERTAPPOINTMENTSLIST]
% make the AWARDS section
\SectionSpacing
\LabelFontstyle{Additional Awards, Positions, Fellowships and Proposals}\\
[INSERTAWARDSLIST]
% make the PUBLICATIONS section
\SectionSpacing
\LabelFontstyle{Publications relevant for the proposal:}\\
[INSERTPUBLICATIONSLIST]
[ENDPERPERSONTEMPLATE]

% COMMENT: The below is the template to use for each EDUCATION entry
[STARTEDUCATION]
[INSERTEDUCATIONYEAR], [INSERTEDUCATIONINSTITUTION], [INSERTEDUCATIONDEGREE], [INSERTEDUCATIONAREA]
[ENDEDUCATION]

% COMMENT: the below is the template for each appointment line in the list
[STARTAPPOINTMENTS]
[INSERTAPPOINTMENTYEAR], [INSERTAPPOINTMENTDESC], [INSERTAPPOINTMENTINSTITUTION]
[ENDAPPOINTMENTS]

% COMMENT: the below is the template for each line in this list
[STARTAWARDS]
[INSERTAWARDYEAR], [INSERTAWARDDESC]
[ENDAWARDS]

% COMMENT: the below is the template used for each publication item in the list
[STARTPUBLICATIONS]
{\PubSym}[INSERTPUB]
[ENDPUBLICATIONS]

% COMMENT: the below are 2 options for how each bio sketch is separated from others
% COMMENT: if the option that each bio sketch be put on a new page (if the purple 
% COMMENT: box above the "biosketch?" column on PERSONNEL & FTE page), then use
% COMMENT: below option. 
[STARTNEWPAGEOPTION]
\newpage 
[ENDNEWPAGEOPTION]

% COMMENT: if the box is not checked, then use this option that writes one bio sketch
% COMMENT: immediately after the other, allowing possibility that a biosketch falls on
% COMMENT: two pages instead of one
[STARTNONEWPAGEOPTION]
\n \medskip \n \hrule \n \vspace{5pt} \n \medskip
[ENDNONEWPAGEOPTION]
}
{%
}
