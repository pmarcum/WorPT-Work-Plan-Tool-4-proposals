%______________________________________
% Instructions and LaTeX code to copy/paste into your document, to incorporate this file into your proposal, now located at:
%
% https://github.com/pmarcum/WorPT-Work-Plan-Tool-4-proposals/blob/main/latexTableTemplates/[INSERTFILENAME]/readme.md
%______________________________________

%
% Template for WorPT script begins below
% (not for general user unless you are planning to alter the WorPT script!)
%

\newif\ifLandScape                 % establish a boolean variable
\def\WorPTfolder{[INSERTFILENAME]} % set the folder name where WorPT files are contained

% ___________________________________________
%       Default formatting preferences
% ___________________________________________
\LandScapefalse             % default is that the landscape mode is OFF

%--- table column widths
\def\TitleWidth{4.1in}      % Title column width
\def\TimelineWidth{1ex}     % skinny timeline columns width
\def\AssignmentsWidth{24ex} % Task assignments column width
\def\SumFteWidth{5ex}       % Total FTE/Sum column width
\def\UnfundedFteWidth{5ex}  % Total FTE/Unfunded column width
\def\FundedFteWidth{5ex}    % Total FTE/Funded column width

%---- fix the Table Number in the caption
\def\TaskAddCounter{-1}   % corrects table number messed up by table,longtable combination)

%---- table compactness controls
\def\SpaceBetweenRows{0.8}    % vertical compactness of rows
\def\SpaceBetweenColumns{1pt} % spacing between columns

%--- control to nudge table to the left or right
\def\NudgeTable{1.5\textwidth}   % adjust to help center table

%---- vertical line colors
\def\TimelineVerticalLineColor{lightgray}
\def\TotalFteVerticalLineColor{lightgray}

%---- preferences of column headers
\def\TaskTimelineHeaderFontsize{\scriptsize}   % Task Timeline label text size
\def\TaskAssignmentsHeaderFontsize{\scriptsize}% Task Assignments label text size
\def\TotalFteHeaderFontsize{\scriptsize}       % Totaled FTEs label text size
\def\YearHeaderFontsize{\scriptsize}           % "YEAR1", "YEAR2"... label text size
\def\TaskTitleHeaderFontsize{\normalsize}      % "TASK TITLES" column label text size
\def\YearSliceHeaderFontsize{\scriptsize}      % "1", "2"... year-quarter label text size
\def\IdWksHeaderFontsize{\normalsize}          % "id wks" text size

%---- symbol preferences for FTE types (sum, funded, unfunded)
\def\TotalFteSumHeaderIcon{\textbf{\large{$\Sigma$}}} % sigma symbol label size
\def\TotalFteUnfundedHeaderIcon{\noDollarIcon{-0.4}{0.4mm}{0.2}{0.15}} % "no dollar" symbol label size
\def\TotalFteFundedHeaderIcon{\dollarIcon{-0.4}{0.2}{0.015}}% green "dollar" symbol label size

%--- preferences of task CATEGORY label
\def\TaskCategoryLabel#1#2{{\normalsize\textbf\scshape{#1}}~{\normalsize\textbf{#2}}} % Task Category label format for row item

%--- preferences of TASK label
\def\TaskTitleLabel#1#2{~{\scriptsize{\textbf{\scshape{#1}}}}~{\color{mediumelectricblue}{\footnotesize\textbf{#2}}}} % Task title format for row item

%--- preferences of timeline color
\def\TimelineColor{mediumelectricblue}          % Task Timeline cell color

%--- preferences for indicating US vs non-US instutional affiliations and funded vs unfunded team members
\def\FundedUsTeam#1{\small \color{blue}{#1}}    % funded US team members ID, \#weeks format
\def\UnfundedUsTeam#1{\small \color{black}{#1}} % UNfunded US team members ID, \#weeks fomat
\def\InternationalTeam#1{\small \color{red}{#1}}% international team members ID, \#weeks format

%--- preferences for formatting the FTE values in far right columns
\def\FteTotalFormat#1{\small{#1}}               % Total FTE/sum value format
\def\FteUnfundedFormat#1{\small{#1}}            % Total FTE/unfunded value format
\def\FteFundedFormat#1{\small \color{blue}{#1}} % Total FTE/funded value format

%--- Format the identity reveals in the table caption
\def\RevealIdentityFormat#1#2{\textbf{#1}: #2}

%--- Delimiter of the task assignments
\def\TaskAssignmentDelimiter{;\hspace{3pt}} % delimiter used in TASK ASSIGNMENTS

%--- internal parameters used by script but not for user
\def\LastYearPlusTwo{[INSERTLASTYEARPLUSTWO]}
\def\NumberYears{[INSERTLASTYEAR]}
\def\SlicesPerYearMinusTwo{[INSERTNUMBERSLICESPERYEARMINUSTWO]} % Num slices/year - first,last slices
\def\TimelineSize{\tiny}                        % width control of TASK TIMELINE cells
\def\RevealIdentities{[INSERTIDTONAMELIST]}

%--- table preamble definition bundled into the "T" variable
\newcolumntype{L}[1]{>{\raggedright\arraybackslash}p{#1}}      % title, assignment columns
\newcolumntype{V}{!{\color{\TimelineVerticalLineColor}\vrule}} % vertical lines in timeline
\newcolumntype{W}{!{\color{\TotalFteVerticalLineColor}\vrule}} % vertical lines in FTE
%
\newcolumntype{T}{
  |L{\TitleWidth}         % title column
  *{\NumberYears}{        % timeline columns
  |p{\TimelineWidth}V*{\SlicesPerYearMinusTwo}{p{\TimelineWidth}V}p{\TimelineWidth}} 
  |L{\AssignmentsWidth}   % task assignment column
  |p{\SumFteWidth}W       % total fte, sum column
  p{\UnfundedFteWidth}W   % total fte, unfunded column
  p{\FundedFteWidth}|     % total fte, funded column
}

\newenvironment{[INSERTFILENAME]}{%
\ifLandScape
  \begin{sidewaystable*}
\else
  \begin{table*}[h!]     % \begin{table} needed to make bottom table caption have same width as table
\fi
\addtocounter{table}{\TaskAddCounter}           % corrects double-counting of longtable and table combination
\renewcommand{\arraystretch}{\SpaceBetweenRows} % controls vertical row separation/spacing
\setlength{\tabcolsep}{\SpaceBetweenColumns}    % controls horizontal column separation/spacing
\centerline{
\begin{minipage}{\NudgeTable}                   % increase to allow a wide table to spill into margin
\begin{longtable}{T}

% ------------------------- First line of headers
\cline{2-[INSERTNUMBERYEARSLICESPLUSONE]}                               
\multicolumn{1}{c|}{} &                                                  % title column
\multicolumn{[INSERTNUMBERYEARSLICES]}{c|}{\TaskTimelineHeaderFontsize{TASK}} & % schedule columns
\multicolumn{1}{c}{} &                                                   % task assignment column
\multicolumn{[INSERTNUMBERFTETYPES]}{c}{}\\                              % 3-column total fte
\cline{[INSERTNUMBERYEARSLICESPLUSTWO]-[INSERTNUMBERYEARSLICESPLUSFIVE]}

% ------------------------- Second line of headers
\multicolumn{1}{c|}{} &                                                    % title column
\multicolumn{[INSERTNUMBERYEARSLICES]}{c|}{\TaskTimelineHeaderFontsize{TIMELINE}} & % schedule columns
\multicolumn{1}{c|}{\TaskAssignmentsHeaderFontsize{TASK}} &                % task assignments column
\multicolumn{[INSERTNUMBERFTETYPES]}{c|}{\TotalFteHeaderFontsize{TOTAL}}\\ % 3-column total fte
\cline{2-[INSERTNUMBERYEARSLICESPLUSONE]}

% -------------------------  Third line of headers
\multicolumn{1}{c|}{}                                                 % title column
[INSERTYEARLABELLIST] &                                               % year label list (YR1, YR2, ...)
\multicolumn{1}{c|}{\TaskAssignmentsHeaderFontsize{ASSIGNMENTS}} &    % task assignments (ID, \#week)
\multicolumn{[INSERTNUMBERFTETYPES]}{c|}{\TotalFteHeaderFontsize{FTE}}\\% 3-column total fte
\cline{1-1}

% COMMENT: below template generates a single year label.  The script glues them together in a single string
% COMMENT: and replaces the INSERTYEARLABELLIST above. 
[STARTYEARLABELTEMPLATE]
 & \multicolumn{[INSERTNUMBERSLICESPERYEAR]}{c|}{\YearHeaderFontsize{YR[INSERTYEARNUMBER]}}
[ENDYEARLABELTEMPLATE]

% -------------------------  Forth line of headers
\multicolumn{1}{|l|}{\TaskTitleHeaderFontsize{TASK TITLES}} % title column
[INSERTYEARSLICELABELLIST] &                                % year slice list (1,2,3,4 1,2,3,4...)
\multicolumn{1}{c|}{\IdWksHeaderFontsize{id~\#wks}} &       % task assignments (ID,\#weeks)
{\TotalFteSumHeaderIcon} &                                  % Col1 Total funded+unfunded FTE
{\TotalFteUnfundedHeaderIcon} &                             % Col2 Total unfunded FTE
{\TotalFteFundedHeaderIcon}                                 % Col3 Total funded FTE
\hline

% COMMENT: below template generates a single year-slice label (for example, a quarter year label ranging from 1 through 4).
% COMMENT: The script glues them together into single string and replaces the above INSERTYEARSLICELABELLIST
[STARTYEARSLICELABELTEMPLATE]
 & \YearSliceHeaderFontsize{[INSERTSLICENUMBER]}
[ENDYEARSLICELABELTEMPLATE]

% begin main contents of table
[INSERTLISTOFCATEGORYSECTIONS]

% COMMENT ****************************** CATEGORY TEMPLATE ***************************************
% COMMENT: the below produces the row for a task category label
% COMMENT: The category sections are put together as a single string delimited by end of line character and then 
% COMMENT: replace the INSERTLISTOFCATEGORYSECTIONS above, constructing the full table
[STARTCATEGORYSECTIONTEMPLATE]
\TaskCategoryLabel{[INSERTCATEGORYLABEL]}{[INSERTCATEGORYTITLE]} % category identification
[INSERTEMPTYYEARSLICESLIST] &  % empty year slices
{} &                           % empty assignments
{} & {} & {}\\                 % empty total fte
[INSERTTASKSUNDERTHISCATEGORY] % tasks under this category
\hline
[ENDCATEGORYSECTIONTEMPLATE]

% COMMENT: the below template makes a single empty year slice. The script figures out how many to make and
% COMMENT: then glues them together into a single string and replaces the INSERTEMPTYYEARSLICESLIST above
[STARTUNCOLOREDTIMELINETEMPLATE]
 & \TimelineSize{}
[ENDUNCOLOREDTIMELINETEMPLATE]

% COMMENT ****************************** ROW TEMPLATE ***************************************
% COMMENT: the below produces a single row for a task under a category. The script then makes several such rows 
% COMMENT: and glues them toegether, delimited by a end of line character, and this single string replaces the 
% COMMENT: INSERTTASKSUNDERTHISCATEGORY above in the CATEGORYSECTIONTEMPLATE above.
[STARTTASKROWTEMPLATE]
% begin task
\TaskTitleLabel{[INSERTTASKLABEL]}{[INSERTTASKTITLE]} % title of this task
[INSERTYEARSLICESLIST] &  % timeline for this task
{[INSERTASSIGNMENTLISTFORTHISTASK]} &  % assignments for this task
\FteTotalFormat{[INSERTFTETOTAL]} & \FteUnfundedFormat{[INSERTFTEUNFUNDED]} & \FteFundedFormat{[INSERTFTEFUNDED]}\\
[ENDTASKROWTEMPLATE] 

% COMMENT: the below template makes a single year slice that falls within the task timeline, helping to construct
% COMMENT: a colored task timeline. The script picks this or the UNCOLORED template above the ROW TEMPLATE written above,
% COMMENT: and puts them together into a single string that replaces INSERTYEARSLICESLIST above
[STARTCOLOREDTIMELINETEMPLATE]
 & \TimelineSize\cellcolor{\TimelineColor}
[ENDCOLOREDTIMELINETEMPLATE]

% COMMENT: the assignment list consists of the team member ids followed by number of work weeks, with delimiter specified in the 
% COMMENT: list of \defs at the top of this file. 
% COMMENT: There are 3 templates, depending on if the team member institutional affiliation is US/funded, US/unfunded, or non-US
% COMMENT: The assignment list is constructed as a single string and then replaces INSERTASSIGNMENTLISTFORTHISTASK in the TASKROWTEMPLATE
% COMMENT: below is format for a team member who is funded and from U.S. institution
[STARTFUNDEDUSTEAMASSIGNMENTTEMPLATE]
\FundedUsTeam{[TEAMID]}\,\FundedUsTeam{[NUMBERWEEKS]}
[ENDFUNDEDUSTEAMASSIGNMENTTEMPLATE]
 
% COMMENT: below is format for a team member who is UNfunded and from U.S. institution
[STARTUNFUNDEDUSTEAMASSIGNMENTTEMPLATE]
\UnfundedUsTeam{[TEAMID]}\,\UnfundedUsTeam{[NUMBERWEEKS]} 
[ENDUNFUNDEDUSTEAMASSIGNMENTTEMPLATE]

% COMMENT: below is format for a team member who is from NON U.S. institution (and by definition, unfunded)
[STARTINTERNATIONALTEAMASSIGNMENTTEMPLATE]
\InternationalTeam{[TEAMID]}\,\InternationalTeam{[NUMBERWEEKS]} 
[ENDINTERNATIONALTEAMASSIGNMENTTEMPLATE]

\end{longtable}
\end{minipage}}}
{%
\ifLandScape
  \end{sidewaystable*}
\else
  \end{table*}
\fi
}

% COMMENT: The caption for this table requires variables (just as the isANONtravel did). The below template defines
% COMMENT: the variables used in the caption formulation that is given in the github help page. Specifically, 
% COMMENT: the below defines the names that go with the abbreviations used in the table.  IF YOU LEAVE OUT 
% COMMENT: THIS INFORMATION FROM THE CAPTIOn, this table could be used as an anonymous one. 
% COMMENT: the below is the template used for each item in the list. The items are delimited by a comma
[STARTIDTONAMETEMPLATE]
\RevealIdentityFormat{[INSERTTEAMID]}{[INSERTTEAMNAME]}
[ENDIDTONAMETEMPLATE]
